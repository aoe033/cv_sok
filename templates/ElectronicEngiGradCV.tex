%% start of file `template.tex'.
%% Copyright 2006-2013 Xavier Danaux (xdanaux@gmail.com).
%
% This work may be distributed and/or modified under the
% conditions of the LaTeX Project Public License version 1.3c,
% available at http://www.latex-project.org/lppl/.


\documentclass[11pt,a4paper,sans, norsk]{moderncv}        % possible options include font size ('10pt', '11pt' and '12pt'), paper size ('a4paper', 'letterpaper', 'a5paper', 'legalpaper', 'executivepaper' and 'landscape') and font family ('sans' and 'roman')

% modern themes
\moderncvstyle{banking}                            % style options are 'casual' (default), 'classic', 'oldstyle' and 'banking'
\moderncvcolor{blue}                                % color options 'blue' (default), 'orange', 'green', 'red', 'purple', 'grey' and 'black'
%\renewcommand{\familydefault}{\sfdefault}         % to set the default font; use '\sfdefault' for the default sans serif font, '\rmdefault' for the default roman one, or any tex font name
%\nopagenumbers{}                                  % uncomment to suppress automatic page numbering for CVs longer than one page

% character encoding
\usepackage[utf8]{inputenc}                       % if you are not using xelatex ou lualatex, replace by the encoding you are using
%\usepackage{CJKutf8}                              % if you need to use CJK to typeset your resume in Chinese, Japanese or Korean

% adjust the page margins
\usepackage[scale=0.75]{geometry}
%\setlength{\hintscolumnwidth}{3cm}                % if you want to change the width of the column with the dates
%\setlength{\makecvtitlenamewidth}{10cm}           % for the 'classic' style, if you want to force the width allocated to your name and avoid line breaks. be careful though, the length is normally calculated to avoid any overlap with your personal info; use this at your own typographical risks...

\usepackage{import}

\usepackage[norsk]{babel}

% personal data
\name{Anders}{Østevik}
\title{Curriculum Vitae}                               % optional, remove / comment the line if not wanted
\address{Fantoftveien 14L, PB 1267, 5075 BERGEN}{}{}% optional, remove / comment the line if not wanted; the "postcode city" and and "country" arguments can be omitted or provided empty
\phone[mobile]{+47 984 92 338}                   % optional, remove / comment the line if not wanted
%\phone[fixed]{01234 123456}                    % optional, remove / comment the line if not wanted
%\phone[fax]{+3~(456)~789~012}                      % optional, remove / comment the line if not wanted
\email{anders.ostevik91@gmail.com}                               % optional, remove / comment the line if not wanted
%\homepage{www.myname.webs.com}                         % optional, remove / comment the line if not wanted
%\extrainfo{additional information}                 % optional, remove / comment the line if not wanted
%\photo[64pt][0.4pt]{picture}                       % optional, remove / comment the line if not wanted; '64pt' is the height the picture must be resized to, 0.4pt is the thickness of the frame around it (put it to 0pt for no frame) and 'picture' is the name of the picture file
%\quote{Some quote}                                 % optional, remove / comment the line if not wanted

% to show numerical labels in the bibliography (default is to show no labels); only useful if you make citations in your resume
%\makeatletter
%\renewcommand*{\bibliographyitemlabel}{\@biblabel{\arabic{enumiv}}}
%\makeatother
%\renewcommand*{\bibliographyitemlabel}{[\arabic{enumiv}]}% CONSIDER REPLACING THE ABOVE BY THIS

% bibliography with mutiple entries
%\usepackage{multibib}
%\newcites{book,misc}{{Books},{Others}}
%----------------------------------------------------------------------------------
%            content
%----------------------------------------------------------------------------------
\begin{document}
%\begin{CJK*}{UTF8}{gbsn}                          % to typeset your resume in Chinese using CJK
%-----       resume       ---------------------------------------------------------
\makecvtitle

\small{Elektronikkingeniør med bachelor i elektronikk fra Høgskolen i Bergen. Arbeider for øyeblikket med min masteroppgave i fysikk ved Universitetet i Bergen.}

\section{Personlig informasjon}

{\renewcommand\labelitemi{}
\begin{itemize}

\item{\textbf{Navn:}	Anders Østevik}
\vspace{3pt}
\item{\textbf{Addresse:}	Fantoftveien 14L, PB 1267, 5075 BERGEN}
\vspace{3pt}
\item{\textbf{Fødselsdato:}	04-07-1991}
\vspace{3pt}
\item{\textbf{Status:}	Samboer, ingen barn}
\vspace{3pt}
\item{\textbf{Mobil:}	98 492 338}
\vspace{3pt}
\item{\textbf{Email:}	anders.ostevik91@gmail.com}

\vspace{6pt}

\end{itemize}
}

\section{Arbeidserfaing}

\vspace{6pt}

\begin{itemize}

\item{\cventry{Juni--Juli 2014, Juni--Juli 2012, Juni--August 2011}{Resepsjonist}{Skudenes Legesenter}{Skudeneshavn, Karmøy}{}{\vspace{3pt}Jeg hadde, sammen med en ansatt resepsjonist, delt ansvar for det administrative i samarbeid med legene, og holdt til i skranken der jeg tok imot pasienter per telefon eller personlig. Jeg hadde oppgaver som organisering av timelister til legene, behandling av bestillinger som pasienttimer, e-resepter og journalutskrifter, sortering av post, samt besvare generelle spørsmål pasientene måtte komme med.
Jeg har ved å jobbe som vikar fått god erfaring i løsing av administrative oppgaver, og har fått arbeidet med mange forskjellige mennesker med ulike behov.}}

\vspace{6pt}

\item{\cventry{Juli--August 2013}{Montør}{Rental Technology \& Service AS}{Åkra, Karmøy}{}{\vspace{3pt}Jeg holdt til i monteringlabben og hadde oppgaver som involverte produksjon og montering
av elektromekaniske komponenter, og produsering og montering av ulike kabelsett for testing av utstyr.}}

\vspace{6pt}

\item{\cventry{Februar--Mai 2013}{Studieassistent i faget ELE102 Datateknikk}{Høgskolen i Bergen}{Bergen}{}{\vspace{3pt}Faget ELE102 gikk ut på å lære programmering i C/C++ og videre bruke språket til å programmere mikrokontrollere. Jeg hadde ansvar for retting og kommentering av innleveringer gjort av studentgrupper bestående av to til tre personer. Jeg hadde også plikt til oppmøte 2 timer i uken, som studiehjelp i et klasserom der studentgruppene kunne be om hjelp til å løse innleveringoppgavene.}}

\end{itemize}

\newpage

\section{Utdanning}

\vspace{5pt}

\subsection{Akademiske kvalifikasjoner}

\vspace{5pt}

\begin{itemize}

\item{\cventry{August 2014-- Juni 2016}{Master i Fysikk}{Universitetet i Bergen}{Bergen}{\textit{Mikroelektronikk}}{}}
\item{\cventry{August 2011-- Juni 2014}{Bachelor i Elektronikk}{Høgskolen i Bergen}{Bergen}{}{}}
\item{\cventry{August 2010-- Mai 2011}{Musikk}{Jæren Folkehøgskole}{Jæren}{}{}}
\item{\cventry{August 2007-- Juni 2010}{Studiespesialisering med realfag}{Kopervik Videregående Skole}{Karmøy}{}{}}

\end{itemize}

\vspace{2pt}

\subsection{Betydelige Prosjekter}

\vspace{5pt}

\begin{itemize}

\item{\textbf{Masteroppgave (Pågående):} \textit{'Firmware Design For the GigaBit Transceiver Common Readout Unit'}

\vspace{3pt}

\small{Selvstendig arbeid over en periode på to semester. Oppgaven gikk ut på å ta del i utviklingen av firmware for en Common Readout Unit (CRU), et utleserkort som skal motta og behandle data fra SLHC (Super Large Hadron Collider), som er i utvikling hos Cern. Oppgaven krevde fordypning i programmeringspråk som C, VHDL og TeX, PCB design, testoppsett for forskjellige forsøk og strukturert skriving med gode kunnskaper i engelsk. Av oppgaven har jeg lært å arbeide strukturert, både selvstendig og med andre mennesker. Jeg har blitt mer erfaren med dokumentering av logg, arbeidsliste og rapporter for gjennomført arbeid.}}

\vspace{6pt}

\item{\textbf{Bacheloroppgave:} \textit{'Testjigg for ClampOn DSP II Strømforsyningskort'}

\vspace{3pt}

\small{Arbeid i gruppe på tre personer over en periode på ett semester. Oppgaven var i samarbeid med bedriften ClampOn og gikk ut på å lage et testoppsett (en såkalt testjigg) for et strømforsyningskort, utviklet av ClampOn. Oppgaven krevde tett samarbeid både med bedrift og internt i gruppen. Som gruppe krevde oppgaven fordypning i analog og digital elektronikk, programmering og kretsutlegg og dokumentering. Av oppgaven har jeg lært å arbeide effektivt i grupper, å fordele oppgaver jevnt og rettferdig, og å dokumentere timelister og rapporter av gjennomført arbeid.}}

\end{itemize}

\section{Tekniske og personline ferdigheter}

\vspace{6pt}

\begin{itemize}

\item \textbf{Programmingspråk:} Dyktig i: C, C++, C\#, Arduino, TeX, VHDL \\ Også grunnleggende evner i: Python, ROOT, SDL, ncurses, Matlab.

\vspace{6pt}

\item \textbf{Software:} Matlab, LTspice, Cadence, Quartus II, SignalTap II, LaTeX, Visual Studio 2010, CodeLite, De fleste Microsoft Office programmer (Word, Excel, Powerpoint, Visio).

\vspace{6pt}

\item \textbf{Generelt:} Jobber bra i team, Imøtekommende, åpen og utadvendt.

\vspace{6pt}

\item \textbf{Andre:} Gode ferdigheter innen lodding av komponenter, Evner til å skrive organiserte og strukturerte rapporter.

\end{itemize}

\newpage

\section{Interesser og utemonfaglige aktiviteter}

\vspace{6pt}

\begin{itemize}

\item{Har interesser innen spillprogrammering og 3D-utvikling. Har siden ungdomskolen hatt ulike spill- og 3D-prosjekter i programmer som \textit{Game Maker: Studio} og \textit{Blender}, og har den siste tiden begynt å programmere spill vha. mediabiblioteket \textit{Simple DirectMedia Layer} i programmeringspråket C.}

\vspace{6pt}

\item{Har en lidenskap for musikk og instrumenter. Jeg har spilt gitar i snart 9 år, selvlært, og har interesse for musikkteori og utvikling av gehør. Jeg har 5 års bakgrunn fra musikkorps der jeg spilte kornett og marsjerte årlig i 17.mai-tog og tok del i sosiale aktiviteter som seminarer og basarer.}

\vspace{6pt}

\item{Har en økende interesse for treverk, og har fullført noen småprosjekter som enkel tre-strengs kassegitar og lyskube. Har den siste tiden gått i anskaffelse av et solid trestykke og skal etter planen påbegynne en elektrisk gitar til sommeren.}

\vspace{6pt}

\item{Jeg er også opptatt av godt kosthold, og spiser variert og sunn mat.}

\end{itemize}

\section{Referanser}

\vspace{6pt}
 
\begin{itemize}

\item{Ta kontakt.}

\end{itemize}


\end{document}


%% end of file `template.tex'.
